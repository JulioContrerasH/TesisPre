% -----------------------------------
% Configuración de Idioma y Fuente
% -----------------------------------
\usepackage[T1]{fontenc}
\usepackage[spanish, es-tabla]{babel} % Configura el idioma a español y reemplaza "cuadro" por "tabla"
\usepackage{newtxtext} % Utiliza la fuente de texto New Times-like


% -----------------------------------
% Gestión de Bibliografía
% -----------------------------------
\usepackage[style=apa, backend=biber]{biblatex}
\addbibresource{3_3_BIBLIOGRAFIA/library.bib} % Carga la bibliografía desde el archivo 'library.bib'
\usepackage{csquotes} % Mejora la presentación de citas en el texto

\setcounter{tocdepth}{3} % Incluye hasta \paragraph en el índice


% -----------------------------------
% Configuración de Documento
% -----------------------------------
\setcounter{secnumdepth}{3} % Numera hasta las subsubsecciones
\decimalpoint % Utiliza punto decimal en lugar de coma
\usepackage{layouts} % Proporciona herramientas para mostrar el diseño del documento
\usepackage{setspace} % Permite ajustar el espaciado entre líneas
\usepackage[margin=1in]{geometry} % Configura los márgenes del documento
 

% -----------------------------------
% Manejo de Imágenes y Tablas
% -----------------------------------
\usepackage{graphicx} % Permite incluir imágenes
\usepackage{changepage} % Facilita la adición de sangrías
\usepackage{float} % Mejora la colocación de objetos flotantes como imágenes
\usepackage{tabularx} % Permite crear tablas con anchos ajustables


% -----------------------------------
% Configuración de Encabezados y Pies de Página
% -----------------------------------
\usepackage{fancyhdr} % Permite personalizar encabezados y pies de página
\pagestyle{fancy} % Aplica el estilo 'fancy' a las páginas


% -----------------------------------
% Otros Ajustes de Estilo y Formato
% -----------------------------------
\usepackage{lipsum} % Genera texto de relleno
\renewcommand{\labelitemi}{$\bullet$} % Define círculos para las viñetas
\usepackage{titlesec} % Personalización de títulos de secciones
\usepackage{tocloft} % Personalización de índices
\usepackage[colorlinks=true,linkcolor=negro,citecolor=negro]{hyperref} % Configura los enlaces dentro del documento


% -----------------------------------
% Configuración de Matemáticas y Diagramas
% -----------------------------------
\usepackage[mathbf=sym]{unicode-math} % Mantiene las fuentes matemáticas
\usepackage{tikz} % Permite crear gráficos y diagramas
\usetikzlibrary{calc,positioning,shapes.geometric,shapes.symbols,shapes.misc} % Carga bibliotecas adicionales para TikZ

% Definiciones de estilos para los elementos de TikZ
\tikzstyle{startstop} = [rectangle, rounded corners, minimum width=3cm, minimum height=0.5cm,text centered, draw=black]
\tikzstyle{io} = [trapezium, trapezium left angle=70, trapezium right angle=110, minimum width=3cm, minimum height=0.5cm, text centered, text width=3cm, draw=black]
\tikzstyle{process} = [rectangle, minimum width=3cm, minimum height=0.5cm, text centered, text width=4cm, draw=black]
\tikzstyle{decision} = [diamond, minimum width=3cm, minimum height=0.5cm, text centered, draw=black]
\tikzstyle{loop} = [chamfered rectangle,chamfered rectangle xsep=2cm,draw=black]
\tikzstyle{arrow} = [->,>=stealth]
\tikzstyle{line}=[draw]


% -----------------------------------
% Soporte para Código Fuente y Color
% -----------------------------------
\usepackage{listings} % Permite incluir código fuente
\usepackage{color} % Permite definir y utilizar colores

% Definición de nuevos colores
\definecolor{codegreen}{rgb}{0,0.6,0}
\definecolor{codegray}{rgb}{0.5,0.5,0.5}
\definecolor{codepurple}{rgb}{0.2,0,1}
\definecolor{codeRojo}{rgb}{0.7,0,0.3}
\definecolor{backcolour}{rgb}{1.0, 1.0, 1.0}


% -----------------------------------
% Configuraciones Adicionales y Personalizaciones
% -----------------------------------
\usepackage{ragged2e} % Proporciona comandos para alinear texto
\usepackage{booktabs} % Mejora la calidad de las tablas
\usepackage{xstring} % Proporciona funciones avanzadas de manipulación de cadenas
\usepackage{multirow} % Mejora la creación de tablas con celdas que abarcan varias filas
\usepackage{array} % Proporciona funciones adicionales para tablas
\usepackage{subcaption} % Permite el uso de subfiguras y subtítulos
\usepackage{chngcntr} % Permite ajustar la numeración de contadores
\counterwithout{figure}{chapter} % Configura la numeración de figuras independientemente de los capítulos
\makeatletter % Comando para reducir errores


% -----------------------------------
% Configuración de Estilo de Listado de Código
% -----------------------------------
% Definición de estilo personalizado para listados de código
\lstdefinestyle{mystyle}{
  backgroundcolor=\color{backcolour},  
  commentstyle=\color{codegreen},
  keywordstyle=\color{codeRojo},
  numberstyle=\tiny\color{codegray},
  stringstyle=\color{codepurple},
  basicstyle=\footnotesize,
  breakatwhitespace=false,         
  breaklines=true,                 
  captionpos=b,                    
  keepspaces=true,                 
  numbers=left,                    
  numbersep=5pt,                  
  showspaces=false,                
  showstringspaces=false,
  showtabs=false,                  
  tabsize=2
}

\lstset{style=mystyle} % Aplica el estilo 'mystyle' a los listados de código


% -----------------------------------
% Ajustes de Formato de Documento
% -----------------------------------
\newenvironment{MyFont}{\fontfamily{ugm}\selectfont}{\par}
\usepackage{changepage} % Agregar espacio a Listing
\renewcommand{\cfttoctitlefont}{\hfill \normalfont\normalsize\bfseries} % Centrado del título del Índice
\renewcommand{\cftaftertoctitle}{\hfill} % Alineación después del título del Índice
\renewcommand{\cftlottitlefont}{\hfill\normalfont\normalsize\bfseries} % Centrado del título de la Lista de Tablas
\renewcommand{\cftafterlottitle}{\hfill} % Alineación después del título de la Lista de Tablas
\renewcommand{\cftloftitlefont}{\hfill\normalfont\normalsize\bfseries} % Centrado del título de la Lista de Figuras
\renewcommand{\cftafterloftitle}{\hfill} % Alineación después del título de la Lista de Figuras


% ------------------------------------
% Espaciado entre Párrafos y Sangría
% ------------------------------------
\setlength{\parindent}{1.27cm} % Configura la sangría
\doublespacing % Aplica doble espaciado en todo el documento
% \usepackage[section]{placeins}


% ------------------------------------
% Cambio del Título de Bibliografía
% ------------------------------------
\addto\captionsspanish{\renewcommand{\bibname}{\centering Referencias bibliográficas}}


% ------------------------------------
% Posicionamiento Vertical de Índices
% ------------------------------------
% Ajustes para el espacio antes y después de los títulos en el Índice, Lista de Figuras y Lista de Tablas
\setlength{\cftbeforelottitleskip}{1pt} % Espacio antes del título en la Lista de Tablas
\renewcommand{\cftafterlottitleskip}{12pt} % Espacio después del título en la Lista de Tablas
\setlength{\cftbeforeloftitleskip}{1pt} % Espacio antes del título en la Lista de Figuras
\renewcommand{\cftafterloftitleskip}{12pt} % Espacio después del título en la Lista de Figuras
\setlength{\cftbeforetoctitleskip}{16pt} % Espacio antes del título en el Índice
\renewcommand{\cftaftertoctitleskip}{12pt} % Espacio después del título en el Índice


% -------------------------------------
% Configuración de Figuras
% -------------------------------------
% Ajustes de la nomenclatura y formato de las figuras
\addto\captionsspanish{\renewcommand{\figurename}{Figura}} % Cambia el nombre de las figuras a "Figura"
\captionsetup{ % Configuración del formato de las etiquetas para figuras
   justification=raggedright, % Alineación a la izquierda
   singlelinecheck=false, % Aplica justificación incluso para textos cortos
   labelsep=period, % Punto después del número de figura
   labelfont=bf % Texto "Figura Número" en negrita
}

% Comando personalizado para insertar figuras
\newcommand{\insertfigure}[3]{
    \begin{figure}[H] % Usa el entorno 'figure' con opción [H] para ubicación exacta
    \caption{\doublespacing \\ \textit{#1}} % Título de la figura con doble espaciado y en cursiva
    \centering % Centra la imagen
    \includegraphics[width=0.7\linewidth]{#2} % Inserta la imagen
    \begin{justify} % Justifica la descripción
        \textit{Nota.} #3 % Descripción o nota asociada a la figura
    \end{justify}
    \label{fig:#1} % Etiqueta para referencia cruzada
    \end{figure}
}


% -------------------------------------
% Configuración de Tablas
% -------------------------------------
\counterwithout{table}{chapter} % Configura la numeración de tablas independiente de capítulos
\addto\captionsspanish{\renewcommand{\tablename}{Tabla}} % Cambia el nombre de las tablas a "Tabla"
\captionsetup[table]{ % Configuración del formato de las etiquetas para tablas
   justification=raggedright, % Alineación a la izquierda
   singlelinecheck=false, % Aplica justificación incluso para textos cortos
   labelsep=period, % Punto después del número de tabla
   labelfont=bf % Texto "Tabla Número" en negrita
}

\newcolumntype{P}[1]{>{\justifying\noindent\arraybackslash}p{#1}} % Nuevo tipo de columna para tablas

% Define que las tablas sean numeradas con números arábigos
\renewcommand{\thetable}{\arabic{table}}  


% -------------------------------------
% Espaciamiento y Nomenclatura en Índices
% -------------------------------------
% Ajustes para el espaciamiento y la nomenclatura en el Índice, Lista de Figuras y Lista de Tablas
\setlength{\cftbeforechapskip}{2mm} % Espaciado antes de capítulos en el Índice
\renewcommand\cftchapafterpnum{\vskip6pt} % Espaciado después de capítulos en el Índice
\renewcommand\cftsecafterpnum{\vskip5pt} % Espaciado después de secciones en el Índice
\renewcommand\cftsubsecafterpnum{\vskip5pt} % Espaciado después de subsecciones en el Índice

\renewcommand{\cftchappresnum}{} % Elimina cualquier prefijo para el número del capítulo
\renewcommand{\cftchapaftersnum}{. } % Añade un punto y un espacio después del número del capítulo
\renewcommand{\cftfigpresnum}{Figura N° } % Prefijo "Figura N°" en la Lista de Figuras
\renewcommand{\cftfigaftersnum}{.} % Separador después del número de figura
\renewcommand{\cftfignumwidth}{6.85 em} % Ancho para el número de figura
\renewcommand{\cfttabpresnum}{Tabla N° } % Prefijo "Tabla N°" en la Lista de Tablas
\renewcommand{\cfttabnumwidth}{6.5 em} % Ancho para el número de tabla


% ------------------------------------
% Cambios en la Numeración de Capítulos
% ------------------------------------
% Cambia el formato de numeración de capítulos a números árabes
\renewcommand{\thechapter}{\arabic{chapter}}

% Cambia el formato de numeración de ecuaciones para incluir el número de capítulo
\renewcommand{\theequation}{\arabic{chapter}.\arabic{equation}} 

% Cambia el formato de numeración de secciones para incluir el número de capítulo
\renewcommand{\thesection}{\arabic{chapter}.\arabic{section}} 


% --------------------------
% Nuevo Contador para Capítulos Especiales
% --------------------------
\newcounter{ChapterRoman}
\renewcommand{\theChapterRoman}{\Roman{ChapterRoman}}


% -----------------------------------
% Configuración de Capítulos Especiales
% -----------------------------------
\newcommand{\Chapter}[1]{
    \clearpage % Asegura que empezamos en una página nueva
    \refstepcounter{ChapterRoman} % Incrementa el contador
    \thispagestyle{empty} % Elimina encabezados y pie de página para esta página
    \null\vfill % Centra el contenido verticalmente
    \begin{center} % Centra el contenido horizontalmente
        \Large\bfseries CAPÍTULO \theChapterRoman\ \\[1em] #1 % Título del capítulo
    \end{center}
    \vfill\null % Centra el contenido verticalmente
    % \addcontentsline{toc}{chapter}{CAPÍTULO \theChapterRoman: #1} % Agrega la entrada al Índice
    \addcontentsline{toc}{chapter}{CAPÍTULO \theChapterRoman} % Elimina ': #1' para quitar el título
    \clearpage % Comienza una nueva página después del título del capítulo
}


% -----------------------------------
% Formateo de Encabezados de Sección
% -----------------------------------
% Nivel 1 (\chapter)
% Para los títulos de capítulos en el cuerpo del documento
\titleformat{\chapter}[block]
  {\normalfont\normalsize\bfseries\centering}
  {\thechapter.} % Elimina el punto si no quieres el número de capítulo seguido de un punto
  {0.5em}
  {\MakeUppercase}
\titlespacing*{\chapter}{0pt}{12pt}{5pt} % Espaciado para \chapter

% Nivel 2 (\section)
\titleformat{\section}[block]
  {\normalfont\normalsize\bfseries\raggedright}
  {\thesection}
  {0.5em}
  {}
\titlespacing*{\section}{0pt}{12pt}{5pt} % Espaciado para \section

% Nivel 3 (\subsection)
\titleformat{\subsection}[block]
  {\normalfont\normalsize\bfseries\itshape\raggedright}
  {\thesubsection}
  {0.5em}
  {}
\titlespacing*{\subsection}{0pt}{12pt}{5pt} % Espaciado para \subsection

% Nivel 4 (\subsubsection)
\titleformat{\subsubsection}[runin]
  {\normalfont\normalsize\bfseries\raggedright}
  {\thesubsubsection}
  {0.5em}
  {}[. \quad]
\titlespacing*{\subsubsection}{1.27cm}{12pt}{0pt} % Espaciado para \subsubsection

% Nivel 5 (\paragraph)
\titleformat{\paragraph}[runin]
  {\normalfont\normalsize\bfseries\itshape\raggedright}
  {\theparagraph}
  {0.5em}
  {}[. \quad]
\titlespacing*{\paragraph}{1.27cm}{12pt}{0pt} % Espaciado para \paragraph


% -------------------------------------
% Definición de Colores
% -------------------------------------
\usepackage{xcolor} % Importación del paquete de colores
\definecolor{granate}{RGB}{113,22,16} % Define el color granate
\definecolor{gris}{RGB}{154,153,157} % Define el color gris
\definecolor{arena}{RGB}{230,217,170} % Define el color arena
\definecolor{azul}{rgb}{0.03, 0.15, 0.4} % Define el color azul
\definecolor{negro}{rgb}{0, 0, 0} % Define el color negro


% -------------------------------------
% Configuración de Listado de Código
% -------------------------------------
\lstset{ % Configuración para listados de código
  language=Python, % Lenguaje de programación
  basicstyle=\ttfamily\small, % Estilo básico
  commentstyle=\color{gray}, % Estilo para comentarios
  keywordstyle=\color{blue}, % Estilo para palabras clave
  stringstyle=\color{red}, % Estilo para cadenas de texto
  numberstyle=\tiny\color{gray}, % Estilo para números de línea
  numbers=left, % Posición de números de línea
  stepnumber=1, % Intervalo de números de línea
  showstringspaces=false, % No mostrar espacios en cadenas
  tabsize=4, % Tamaño de tabulación
  breaklines=true, % Romper líneas largas
  frame=single, % Marco alrededor del código
  literate=% Configuración para caracteres especiales
  {á}{{\'a}}1 {é}{{\'e}}1 {í}{{\'i}}1 {ó}{{\'o}}1 {ú}{{\'u}}1
  {Á}{{\'A}}1 {É}{{\'E}}1 {Í}{{\'I}}1 {Ó}{{\'O}}1 {Ú}{{\'U}}1
  {ñ}{{\~n}}1 {Ñ}{{\~N}}1 {ü}{{\"u}}1 {Ü}{{\"U}}1
}