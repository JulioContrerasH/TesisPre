\Chapter{}
\chapter{Presupuesto}
    \begin{table}[H]
        \caption{\doublespacing \\ \textit{Presupuesto para el proyecto de investigación.}}
        \begin{spacing}{8}
            \fontsize{8pt}{2pt}\selectfont  
            \begin{tabularx}{\linewidth}{*{4}{P{4.5cm}}} 
                \toprule
                \multicolumn{4}{c}{\textbf{Detalles del presupuesto total (Montos aproximados en S/.)}} \\ 
                \midrule
                \textbf{Rubro} & \textbf{Costo unitario} & \textbf{Cantidad} & \textbf{Total} \\
                \midrule
                1. Equipos: & & & \\ 
                - Laptop & 3000 & 1 & 3000 \\ 
                - Disco duro de 1 Tbyte & 200 & 1 & 200 \\ 
                2. Internet & & & \\ 
                - Servicio de internet por 12 meses & 1400 & 1 & 1400 \\
                3. Papelería y útiles & & & \\
                - Materiales de escritorio & 150 & 1 & 150 \\ 
                - Impresiones & 800 & 1 & 800 \\ 
                4. Software & & & \\ \
                - QGIS & 0 & 1 & 0 \\ 
                - Python & 0 & 1 & 0 \\ \
                - R & 0 & 1 & 0 \\ 
                \bottomrule
            \end{tabularx}
        \end{spacing}
        \vspace{1\baselineskip}
        \textit{Nota.} El presupuesto de investigación incluye costos de equipamiento como laptop y disco duro, gastos anuales de internet, material de oficina, impresiones, y uso de software gratuito como QGIS, Python y R, detallando montos unitarios y costos totales por ítem.
        \label{Presupuesto}
    \end{table}